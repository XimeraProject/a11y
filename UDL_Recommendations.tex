\documentclass{ximera}
\usepackage{tikz}
\usepackage{color}
\usepackage{amsmath,amssymb,amsfonts}
%\usepackage{helvet}
%\renewcommand{\familydefault}{\sfdefault}
\author{Jeffrey Kuan, Ph.D., CPACC, ADS}
\usepackage{todonotes}

\renewcommand{\alt}[1]{\ensuremath{\vadjust{\todo{#1}}}}

\makeatletter
\define@key{answer}{alt}{}
\makeatother

\let\image\relax
\let\endimage\relax
\NewEnviron{image}{
  \begin{center}\BODY\end{center}% center
}

\colorlet{textColor}{black}
\colorlet{background}{white}
\colorlet{penColor}{blue!50!black} % Color of a curve in a plot
\colorlet{penColor2}{red!50!black}% Color of a curve in a plot
\colorlet{penColor3}{red!50!blue} % Color of a curve in a plot
\colorlet{penColor4}{green!50!black} % Color of a curve in a plot
\colorlet{penColor5}{orange!80!black} % Color of a curve in a plot
\colorlet{penColor6}{yellow!70!black} % Color of a curve in a plot
\colorlet{fill1}{penColor!20} % Color of fill in a plot
\colorlet{fill2}{penColor2!20} % Color of fill in a plot
\colorlet{fillp}{fill1} % Color of positive area
\colorlet{filln}{penColor2!20} % Color of negative area
\colorlet{fill3}{penColor3!20} % Fill
\colorlet{fill4}{penColor4!20} % Fill
\colorlet{fill5}{penColor5!20} % Fill
\colorlet{gridColor}{gray!50} % Color of grid in a plot
 %% Loads the graphics path
\title{UDL Recommendations for Ximera}
\license{CC: 0}
\begin{document}
 
\begin{abstract}
   I will present recommendations for UDL (Universal Design in Learning) for Ximera.
\end{abstract}
\maketitle
%\part{Introduction}
%\chapterstyle
%    \activity{basics/basicWorksheet}
%    \sectionstyle
%    \activity{basics/exercises/someExercises}
 
%    \chapterstyle
%    \activity{basics/graphicsInteractives}
 
Accessibility statement: This webpage is WCAG2.1AA, thanks to Tailor Swift Bot. It was manually
tested on NVDA. 
 
\includegraphics[width=10cm,height=7cm,alt="Company Logo of Tailor Swift Bot"]{TSBotsLogo.png}
 
\section{Introduction}

The UDL Guidelines were developed by CAST \cite{CAST} (Center for Applied Special Technology), and are 
 ``a framework developed ... to improve and optimize teaching and learning for all people based on scientific insights into how humans learn.''
The current version, UDL Guidelines 3.0, were released on July 30th, 2024. As part of my CPACC (Certified Professional in Accessibility Core Competencies) 
credentials, I studied UDL Guidelines 2.2. Since the release of UDL 3.0, I completed 18 hours of Continuing Accessibility Education Credits using the
CAST modules located at \cite{NCAEM}. Additionally, I studied the documents highlighting the differences between UDL 2.2 and UDL 3.0.

The guidelines are categorized into several categories:

\begin{itemize}
\item Multiple Means of Engagement, the “why” of learning. For example, a student in a math class may ask \underline{why} they are learning the material.
In mathematical parlance, this is often asking for the ``motivation'' behind the math, whether it is theory or to solve a real--world problem.
If students understand why something is being taught to them, they are more likely to be engaged. 
\item Multiple Means of Representation, the “what” of learning. Here, the ``what'' refers to \underline{what} the students are learning. For example, a certain mathematical subject can be presented to the
students in multiple ways. Considering the example of integration, it is usually presented both as anti--differentiation and the area under the graph of
the function. 
\item Multiple Means of Action and Expression, the “how” of learning. This refers to \underline{how} students can express their knowledge to the
instructor, usually in the form of homework assignments, projects, or examinations. 
\end{itemize}

Mathematics education poses unique challenges for UDL. Below, I elaborate on a few examples. 


Due to mathematics being both ``pure'' and ``applied'', it can be difficult to engage the students. As a concrete example, a probability class
could have students from both a pure mathematical background, as well as statistics and engineering students. The formula for the expectation
(or mean) of a real--valued continuous random variable
\[
\mathbb{E}[X] = \mu_X = \int_{\mathbb{R}} x f_X(x)dx
\]
has a natural real--world application. For instance, if you pick a real number between 0 and 1 uniformly at random, the formula says the mean is
\[
\int_0^1 xdx = \frac{1}{2},
\]
which matches most people's intuition. However, a pure mathematician may prefer to understand this formula in the context of measure theory, where it would 
read as
\[
\int_{\Omega} X d\mathbb{P}. 
\]
These are quite different approaches and motivations, and it can be difficult to motivate students of different backgrounds. 

To adhere to ``multiple means of representation'', the most common methods involve using a mathematical statement with symbols, a picture, and an intuitive
description in words. For certain levels of mathematics, this can be difficult becase some concepts either do not have intuitive descriptions or can not
be visualized (the famous quote about the equivalence of Axiom of Choice is an example). As an example, we can consider this image:

\begin{figure}
\includegraphics[alt="a bunch of squiggles"]{mumforddrawing.jpg}
\caption{Mumford's drawing of \( Spec(Z[x]) \) }
\end{figure}

Because I am not an algebraic geometer, I cannot describe this image, nor could an instructional designer. However, here is some alternative text provided 
by someone else:

\begin{blockquote}
There are straight vertical lines with one for each prime number. The first four have been labelled V((p)) for the respective primes p. Each vertical line ends in a squiggle at the top which is labelled with its generic point [(p)].
There are several curves going across the diagram representing vanishing sets of irreducible polynomials. Each of these curves ends in a squiggle at the right which is labelled with its generic point [(f)] for the respective irreducible polynomial.
These curves with generic point [(f)] intersect with the vertical lines at various points labelled by [(p,g)] for the corresponding prime p and an irreducible polynomial (g), such that f can be generated by p and g.
There is a big squiggle labelled ``[(0)] generic point''.
\end{blockquote}

``Multiple means of action and expression'' is difficult in math, because in lower--level courses there is usually only one right answer. However, 
there is some leeway for showing partial credit. For proof--based classes, there is more flexibility. 

In this document, I will outline my UDL suggestions for Ximera, organized by guidelines. 

\section{Engagement}

\subsection{Motivations}

An author using Ximera should have the ability to designate certain motivations with either ``pure''
or ``applied'' (or other options, depending on the author). Students can then skip through motivations
that are not interesting to them. A LaTeX command that creates accordions would help clarify to the 
student what can be safely skipped. 


\section{Representation}

\subsection{Intent Tag}

One of the most natural ways to represent mathematics in multiple modes is with speech. Most instructors
do this instictively, by reading the text that they are writing during a lecture. In web accessibility, 
this is commonly called ``alternative text.'' However, I would recommend a different terminology for
several reasons. 

First, one of the updates from UDL 3.0 is to avoid the phrase ``alternative'' as it implies that one 
format is ``standard.'' Second, the phrase ``alternative text'' is often used in the context of images, 
which leads many authors to insert images of math equations, which is outdated mode of representation 
\cite{PSU}. Third, MathML4 will release an ``intent'' environment which allows for the author's intent
for how a math expression should be spoken. 

My recommendation is for Ximera to include a command or environment in LaTeX, called ``intent'', 
which will allow the author to insert their intended representation of the math expression in speech.
Neil Soiffer, one of the main developers of MathML, has also 
suggested creating a PDF/UA as an intermediary to MathML.

\subsubsection{MathML related suggestions}

Additionally, Peter Korn at Amazon suggested SSML for more natural sounding text--to--speech, with better
prosody. He also suggested that the reader have the option to specify verbosity of alternative text. 
Also, in the event that content is in a language other than English, Ximera should reach out 
to me to check on the MathML--to--Braille conversion, since Nemeth Braille is specific to English. 

\subsection{Spacing and Fonts}

This recommendation is more specific for dyslexic readers. The British Dyslexia Association has 
style guidelines \cite{BD}. In the context of mathematics, there are unique challenges, as some evidence 
suggests that some readers prefer serif fonts for math symbols. The ``ideal'' recommendation is a setting
that allowed the reader to choose their own font size, tracking and line spacing. An acceptable recommendation
is to prevent authors from creating content that is not dyslexia--friendly (either with a CSS file or
by altering the Ximera document class). 

There is also a Swedish Dyslexia Institution which released guidelines, although it is in Swedish. 
I can request a translation if desired. 





\subsection{Videos}

Ximera has a command to link to YouTube videos. Of course, the videos should have accurate captioning. 
Additionally, the transcription should be downloadable (this is in WCAG2.1AA, section 1.2.3). 
I would recommend a Ximera command that lets the reader download the transcription. 

\subsection{Accessibility Statements}

An important feature of the above suggestions is to make the reader aware of different acessibility features.
Authors using Ximera should have the ability to insert a stanard accessibility statement (see \cite{W3C} 
for suggestions), with a designated contact person. Of course, authors can also refer to their home 
institutions as well. 

\section{Action and Expression}

\subsection{Answer Environment}

Ideally, students could submit their written work and receive partial credit. So for example, 
a student who gets an answer wrong, but with a common mistake, can be informed of this possibility. 
If possible, the answer environment should give more detailed feedback. Also, as an accessiblity 
requirement, we should check that the form works with assistive technology (speech--to--text, etc.)





\begin{thebibliography}{99}
   \bibitem{CAST} CAST (2024). Universal Design for Learning Guidelines version 3.0. Retrieved from \url{https://udlguidelines.cast.org}
   \bibitem{NCAEM} CAST. National Center on Accessible Educational Materials. https://aem.cast.org/learning-series/online-learning-series-accessible-materials-technologies
   \bibitem{UDL} CAST. https://www.cast.org/contact/udlg-docs/
   \bibitem{PSU} Penn State University. https://accessibility.psu.edu/math/equations/
   \bibitem{BD} British Dyslexia Association. https://www.bdadyslexia.org.uk/advice/employers/creating-a-dyslexia-friendly-workplace/dyslexia-friendly-style-guide
   \bibitem{W3C} World Wide Web Consortium. https://www.w3.org/WAI/planning/statements/
\end{thebibliography}

\end{document}